\documentclass[watermark]{pbpreprint}
\usepackage{listings}
\usepackage{bold-extra}

\newsubfloat{table}

\begin{document}
\title{Storing and transferring files}
\newcommand{\subtitle}{Big data i biovetenskap}
\renewcommand{\maketitlehookb}{\centering\textsc{\subtitle}}
\author{Jonathan Alvarsson}
\maketitle
\begin{KeepFromToc}
   \tableofcontents
\end{KeepFromToc}

\lstset{basicstyle=\ttfamily\small, frame = single}

\section{Introduction}
In this lab we will try out different approaches to transferring and storing
files. Our files will not quite qualify as being big data instead we will focus
on getting a chance to test different tools. Much of what we will do will be to
copy files from one location to another. Although you are free to use whatever
locations you want I will suggest that you work inside your assigned Kubernetes
pods (also known as Jupyter notebooks) on the cluster by accessing a terminal
there and then copy things to your Uppmax account. This is the best way because
your pods are not configured to accept any connections from the outside whereas
the Uppmax systems are.

\section{Downloading from the Zink database using \texttt{wget}}
Zink is ``a freely available database of commercially-available compounds for
virtual screening'', we will be using the 2012 version and download some
datasets from it. On \url{http://zinc12.docking.org/subsets/lead-like} there
are a lot of datasets listed if you click on the \texttt{Downloads} tab. The
molecules can be downloaded in different formats. Let's start by grabbing all
of them in the SMILES format using \texttt{wget}:
\begin{lstlisting}[language=bash]
$ wget http://zinc12.docking.org/db/bysubset/1/1_p0.smi.gz
\end{lstlisting}
We have now downloaded a gzipped text file. We can have a look at it using
\texttt{zless}:
\begin{lstlisting}[language=bash]
$ zless 1_p0.smi.g
\end{lstlisting}
As you can see the file contains two columns. The first column is the SMILES
string and the second column is an identifier. The simplified molecular-input
line-entry system (SMILES) is a way to write down a molecular structure as a
string. We can also unpack it using \texttt{gunzip}:
\begin{lstlisting}[language=bash]
$ gunzip 1_p0.smi.gz
\end{lstlisting}

\section{Transfer file using \texttt{scp}}

\section{Transfer file using SFTP}

\section{Synchronize folder with \texttt{rsync}}
\end{document}
