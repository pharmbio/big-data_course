\documentclass[aspectratio=169,usenames,dvipsnames]{beamer}

\usepackage{pgf}  
\usepackage{tikz}
\usetikzlibrary{arrows}
\usepgflibrary{shapes.arrows} 
\usetikzlibrary{intersections}
\usetikzlibrary{calc}
\usetikzlibrary{fit}
\usetikzlibrary{automata,positioning}
\usepackage{pgfplots,stackengine}
\usepackage{fontspec}
\usepackage{fancyvrb}
\usepackage{wasysym}
\usepackage{unicode-math}
\usepackage{import}
\usepackage{rotating}
\usepackage{gensymb}
\usepackage{chemfig}
\usepackage{rotating}
\usepackage{booktabs}
\usepackage{pifont}
\usepackage{wrapfig}
\usepackage{mathtools}
\usepackage{graphbox}
\usepackage{epigraph}
\usepackage{listings}
\usepackage{verbatim}
\usepackage{hologo}
\usepackage{wrapfig}
\usepackage[absolute,overlay]{textpos}
\usepackage[euler-digits,euler-hat-accent]{eulervm}
%\logo{\pgfputat{\pgfxy(.45,.5)}{\pgfbox[center]{\includegraphics[width=1.7cm]{Figures/uu_shadow.pngu}}}}

\usetheme{Copenhagen}
\usecolortheme{beaver}

\definecolor{uured}{RGB}{153,0,0}
\setbeamercolor{block title}{use=structure,fg=white,bg=uured}
\setbeamercolor*{item}{fg=red}

\newcommand{\unilogo}{
  \setlength{\TPHorizModule}{1pt}
  \setlength{\TPVertModule}{1pt}
  \begin{textblock}{1}(26,-10)
   \includegraphics[height=70pt, align=c]{Figures/uu_shadow.png}
  \end{textblock}
  } 

\pgfmathdeclarefunction{gauss}{2}{%
  \pgfmathparse{1/(#2*sqrt(2*pi))*exp(-((x-#1)^2)/(2*#2^2))}%
}
  
\makeatletter
    \newcases{mycases}{\quad}{%
        \hfil$\m@th\displaystyle{##}$}{$\m@th\displaystyle{##}$\hfil}{\lbrace}{.}
\makeatother

\addtobeamertemplate{frametitle}{}{%
    \unilogo
}
\LetLtxMacro{\oldBlock}{\block}
\LetLtxMacro{\oldEndBlock}{\endblock}
\renewcommand{\block}{\begin{center}\begin{minipage}{0.8\textwidth}\oldBlock}
\renewcommand{\endblock}{\oldEndBlock\end{minipage}\end{center}}
\definecolor{darkpastelgreen}{rgb}{0.01, 0.75, 0.24}

\setlength{\fboxsep}{0pt}

\begin{document}
\graphicspath{{Figures/}}
\setsansfont[ItalicFont = Optima Italic,
             BoldFont = Optima Bold,
             Ligatures=TeX ]
            {Optima Regular}
\setmainfont[ItalicFont = Optima Italic,
             BoldFont = Optima Bold,
             Ligatures=TeX]
            {Optima Regular}
\newfontfamily\commentfont[]{Chalkboard}
\newfontfamily\DejaSans{DejaVuSans.ttf}
\newfontfamily\herculanum[]{Herculanum}
\newfontfamily\timesfont[ItalicFont = Times New Roman Italic]{Times New Roman}
\newcommand{\lmr}{\fontfamily{lmr}\selectfont}
\newfontfamily\zA[Ligatures={Common, Rare}, Variant=1] {Zapfino}
\newfontfamily\zB[Ligatures={Common, Rare}, Variant=2] {Zapfino}
\newfontfamily\zC[Ligatures={Common, Rare}, Variant=3] {Zapfino}
\newfontfamily\zD[Ligatures={Common, Rare}, Variant=4] {Zapfino}
\newfontfamily\zE[Ligatures={Common, Rare}, Variant=5] {Zapfino}
\newfontfamily\zF[Ligatures={Common, Rare}, Variant=6] {Zapfino}
\newfontfamily\zG[Ligatures={Common, Rare}, Variant=7] {Zapfino}
\renewcommand\UrlFont{\color{blue}}
\renewcommand\thefootnote{\textcolor{uured}{\arabic{footnote}}}
\setbeamercolor{alerted text}{fg=uured}
\lstset{basicstyle=\ttfamily\scriptsize, frame=single }
\newcommand{\TikZ}{{\lmr Ti\textit{k}Z}}

\title{An introduction to Apache Spark}   
\author{Jonathan Alvarsson} 
%\titlegraphic{\vfill\includegraphics[width=18em]{Figures/ORN_large.png}}
\date{\today} 

\setbeamertemplate{background}{%
    \parbox[c][\paperheight]{\paperwidth}{%
        \vfill
        \hfill
        \includegraphics[height=0.65\textheight]{Figures/sigill.png}
    }   
}
\begin{frame}[plain]
\unilogo \vspace{1cm} \titlepage
\begin{tikzpicture}[remember picture,overlay]
\tikz[remember picture, overlay] \fill[uured] (current page.north west) rectangle ++(\paperwidth,-0.5cm);
\end{tikzpicture}%
\end{frame}

\setbeamertemplate{background}{}
\renewcommand{\unilogo}{
  \setlength{\TPHorizModule}{1pt}
  \setlength{\TPVertModule}{1pt}
  \begin{textblock}{1}(0,0)
   \includegraphics[height=27pt, align=c]{Figures/uu.png}
  \end{textblock}
  } 
\section{Background}
    \begin{frame}
    \frametitle{Outline}
    \begin{minipage}{0.25\textwidth}
    \mbox{}
    \end{minipage}
    \begin{minipage}{0.6\textwidth}
    \tableofcontents[hideallsubsections]
    \end{minipage}
    \end{frame}

    \subsection{LISP}
\setbeamertemplate{background}{%
    \parbox[c][\paperheight]{\paperwidth}{%
        \vskip -0.1 ex \hskip -1 em
        \includegraphics[width=1.05\paperwidth]{Figures/LISP.png}
    }   
}
\begin{frame}[plain]
        %\vskip -20 ex
    %\Large\qquad\qquad\qquad\color{black} {\zA In the beginning there was} \\
    %\Huge LISP
\hfill        {\small\color{white}IBM 704 Computer in 1957}
\vfill\vfill\vfill\vfill\vfill
\begin{minipage}[c]{0.5\textwidth}
        \pause
        \oldBlock{\zA "In the beginning there was":}
        \begin{center}
        \vspace*{2ex}
        \Huge LISP
        \vspace*{1ex}
        \end{center}
        \oldEndBlock
\end{minipage}\hfill


\end{frame}
\setbeamertemplate{background}{}
    \begin{frame}
        \frametitle{LISP}
        \framesubtitle{LISt Processor}
            \begin{center}\texttt{(print "Hello, World!")}\end{center}
        \begin{block}{LISP}
            \begin{itemize}
                \item created in 1958
                \item is the second-oldest high-level programming language in widespread use today (the oldest being FORTRAN)
            \end{itemize}
        \end{block}

    \end{frame}
    \begin{frame}[fragile]
        \frametitle{LISP}
        \framesubtitle{LISt Processor}
\begin{minipage}{0.42\textwidth}
\raggedright
LISP is a powerful, but in my opinion, somewhat special language. Here is an example\footnotemark calculating the factorial of a number {\small (directly from Wikipedia)}:
\begin{lstlisting}
(defun factorial (n)
    (if (= n 0) 1
        (* n (factorial (- n 1)))))
\end{lstlisting}
\end{minipage}
\hfill
\uncover<2->{
\begin{minipage}{0.55\textwidth}
\centering
\includegraphics[width=1\textwidth]{Figures/lisp_cycles.png}
\small \texttt{https://xkcd.com/297/}
\end{minipage}}
\footnotetext{You do \alert{\underline{not}} need to learn LISP in this course}
    \end{frame}

    \subsection{Map and Reduce}

    \begin{frame}
        \frametitle{Map and Reduce}
        \color{lightgray}
        \begin{center}
            \only<1->{\only<1>{\color{black}}So, why do I talk so much about LISP?}
        \end{center}
        \begin{center}
            \only<2->{\only<2>{\color{black}}It's because its list-based structure lent itself well for \texttt{map} and \texttt{reduce}}
        \end{center}
        \begin{center}
            \only<3->{\only<3>{\color{black}}Which later gave rise to the MapReduce programming model}
        \end{center}
        \begin{center}
            \only<4->{\only<4>{\color{black}}but first, let's talk about \texttt{map} and \texttt{reduce}\ldots}
        \end{center}
    \end{frame}

    \begin{frame}[fragile]
        \frametitle{Map and Reduce}
        \framesubtitle{Map}

            \begin{block}{Map}
                \texttt{map} is a higher-order function that applies a given
                function to each element of, \textit{e.g.}, a list. It can also
                be called \texttt{apply-to-all}.
            \end{block}
            \begin{uncoverenv}<2->
            \begin{block}{Example: \hfill \small(Python)}
            \begin{center}
            \begin{minipage}{0.8\linewidth}
            \begin{lstlisting}
square = lambda x:x*x
a = [1,2,3,4]
list(map(square, a))
            \end{lstlisting}
            $\Rightarrow$ \texttt{[1, 4, 9, 16]}
            \end{minipage}
            \end{center}
            \end{block}
            \end{uncoverenv}
        
    \end{frame}
    
    \begin{frame}[fragile]
        \frametitle{Map and Reduce}
        \framesubtitle{Reduce}

            \begin{block}{Reduce}
                \texttt{reduce} is a higher-order function that by using a
                given function combines the part of, \textit{e.g.}, a list into
                a return value.
            \end{block}
            \begin{uncoverenv}<2->
            \begin{block}{Example: \hfill \small(Python)}
            \begin{center}
            \begin{minipage}{0.8\linewidth}
            \begin{lstlisting}
from functools import reduce
sum = lambda a,b:a+b
a = [1,2,3,4]
reduce(sum,a)
            \end{lstlisting}
            $\Rightarrow$ 10
            \end{minipage}
            \end{center}
            \end{block}
            \end{uncoverenv}
    \end{frame}

    \begin{frame}
        \frametitle{Map and Reduce}
        \framesubtitle{Map and Reduce}
        \begin{block}{Map and Reduce}
    \centering
    \begin{tikzpicture}[node distance=0]
      \only<1>{
        \node (a)  [draw,circle, minimum size=8pt]{};
        \node (b)  [right of = a, node distance = 16 pt, draw,circle, minimum size=8pt]{};
        \node (c)  [right of = b, node distance = 16 pt, draw,circle, minimum size=8pt]{};
        \node (d)  [right of = c, node distance = 16 pt, draw,circle, minimum size=8pt]{};
        \node (e)  [right of = d, node distance = 16 pt, draw,circle, minimum size=8pt]{};
        \node (f)  [right of = e, node distance = 16 pt, draw,circle, minimum size=8pt]{};
        \node (g)  [right of = f, node distance = 16 pt, draw,circle, minimum size=8pt]{};
        \node (h)  [right of = g, node distance = 16 pt, draw,circle, minimum size=8pt]{};
        \node (i)  [right of = h, node distance = 16 pt, draw,circle, minimum size=8pt]{};
        \node (j)  [right of = i, node distance = 16 pt, draw,circle, minimum size=8pt]{};
        
        \node (aa)  [below of = a, node distance = 32 pt, draw,circle, minimum size=8pt]{};
        \node (ab)  [right of = aa, node distance = 16 pt, draw,circle, minimum size=8pt]{};
        \node (ac)  [right of = ab, node distance = 16 pt, draw,circle, minimum size=8pt]{};
        \node (ad)  [right of = ac, node distance = 16 pt, draw,circle, minimum size=8pt]{};
        \node (ae)  [right of = ad, node distance = 16 pt, draw,circle, minimum size=8pt]{};
        \node (af)  [right of = ae, node distance = 16 pt, draw,circle, minimum size=8pt]{};
        \node (ag)  [right of = af, node distance = 16 pt, draw,circle, minimum size=8pt]{};
        \node (ah)  [right of = ag, node distance = 16 pt, draw,circle, minimum size=8pt]{};
        \node (ai)  [right of = ah, node distance = 16 pt, draw,circle, minimum size=8pt]{};
        \node (aj)  [right of = ai, node distance = 16 pt, draw,circle, minimum size=8pt]{};

        \draw [->, shorten <=3pt, shorten >=3pt] (a) to (aa);
        \draw [->, shorten <=3pt, shorten >=3pt] (b) to (ab);
        \draw [->, shorten <=3pt, shorten >=3pt] (c) to (ac);
        \draw [->, shorten <=3pt, shorten >=3pt] (d) to (ad);
        \draw [->, shorten <=3pt, shorten >=3pt] (e) to (ae);
        \draw [->, shorten <=3pt, shorten >=3pt] (f) to (af);
        \draw [->, shorten <=3pt, shorten >=3pt] (g) to (ag);
        \draw [->, shorten <=3pt, shorten >=3pt] (h) to (ah);
        \draw [->, shorten <=3pt, shorten >=3pt] (i) to (ai);
        \draw [->, shorten <=3pt, shorten >=3pt] (j) to (aj);

        \node (startbrace1) [node distance = 10 pt, above right of = j] {};
        \node (stopbrace1) [node distance = 10 pt, below right of = aj, yshift=4pt] {};
        \draw [decorate,decoration={brace,amplitude=10pt,raise=4pt}]
              (startbrace1) -- (stopbrace1) node [black,midway,xshift=1.05cm] {Map};
      }
      \uncover<3->{
        \node (a)  [draw,circle, minimum size=8pt]{};
        \node (b)  [right of = a, node distance = 16 pt, draw,circle, minimum size=8pt]{};
        \node (c)  [right of = b, node distance = 16 pt, draw,circle, minimum size=8pt]{};
        \node (d)  [right of = c, node distance = 16 pt, draw,circle, minimum size=8pt]{};
        \node (e)  [right of = d, node distance = 16 pt, draw,circle, minimum size=8pt]{};
        \node (f)  [right of = e, node distance = 16 pt, draw,circle, minimum size=8pt]{};
        \node (g)  [right of = f, node distance = 16 pt, draw,circle, minimum size=8pt]{};
        \node (h)  [right of = g, node distance = 16 pt, draw,circle, minimum size=8pt]{};
        \node (i)  [right of = h, node distance = 16 pt, draw,circle, minimum size=8pt]{};
        \node (j)  [right of = i, node distance = 16 pt, draw,circle, minimum size=8pt]{};
      }
       
      \only<2>{
        \node (aa)  [below of = a, node distance = 32 pt, draw,circle, minimum size=8pt]{};
        \node (ab)  [right of = aa, node distance = 16 pt, draw,circle, minimum size=8pt]{};
        \node (ac)  [right of = ab, node distance = 16 pt, draw,circle, minimum size=8pt]{};
        \node (ad)  [right of = ac, node distance = 16 pt, draw,circle, minimum size=8pt]{};
        \node (ae)  [right of = ad, node distance = 16 pt, draw,circle, minimum size=8pt]{};
        \node (af)  [right of = ae, node distance = 16 pt, draw,circle, minimum size=8pt]{};
        \node (ag)  [right of = af, node distance = 16 pt, draw,circle, minimum size=8pt]{};
        \node (ah)  [right of = ag, node distance = 16 pt, draw,circle, minimum size=8pt]{};
        \node (ai)  [right of = ah, node distance = 16 pt, draw,circle, minimum size=8pt]{};
        \node (aj)  [right of = ai, node distance = 16 pt, draw,circle, minimum size=8pt]{};
      }
      \uncover<3->{
        \draw [->, shorten <=3pt, shorten >=3pt] (a) to (aa);
        \draw [->, shorten <=3pt, shorten >=3pt] (b) to (ab);
        \draw [->, shorten <=3pt, shorten >=3pt] (c) to (ac);
        \draw [->, shorten <=3pt, shorten >=3pt] (d) to (ad);
        \draw [->, shorten <=3pt, shorten >=3pt] (e) to (ae);
        \draw [->, shorten <=3pt, shorten >=3pt] (f) to (af);
        \draw [->, shorten <=3pt, shorten >=3pt] (g) to (ag);
        \draw [->, shorten <=3pt, shorten >=3pt] (h) to (ah);
        \draw [->, shorten <=3pt, shorten >=3pt] (i) to (ai);
        \draw [->, shorten <=3pt, shorten >=3pt] (j) to (aj);
      }

      \only<2>{
        \node (startbrace1) [node distance = 10 pt, above right of = j] {};
        \node (stopbrace1) [node distance = 10 pt, below right of = aj, yshift=4pt] {};
        
        \node (ba)  [below of = aa, node distance = 24 pt, xshift=8pt, draw, circle, minimum size=8pt]{};
        \node (bb)  [below of = ac, node distance = 24 pt, xshift=8pt, draw, circle, minimum size=8pt]{};
        \node (bc)  [below of = ae, node distance = 24 pt, xshift=8pt, draw, circle, minimum size=8pt]{};
        \node (bd)  [below of = ag, node distance = 24 pt, xshift=8pt, draw, circle, minimum size=8pt]{};
        \node (be)  [below of = ai, node distance = 24 pt, xshift=8pt, draw, circle, minimum size=8pt]{};
        
        \node (ca)  [below of = ba, node distance = 24 pt, xshift=16pt, draw, circle, minimum size=8pt]{};
        \node (cb)  [below of = bc, node distance = 24 pt, xshift=16pt, draw, circle, minimum size=8pt]{};
        
        \node (da)  [below of = ca, node distance = 36 pt, xshift=48pt, draw, circle, minimum size=8pt]{};
        \node (db)  [below of = bd, node distance = 36 pt, xshift=16pt, draw, circle, minimum size=8pt]{};
        
        \draw [->, shorten <=3pt, shorten >=3pt] (aa) to (ba);
        \draw [->, shorten <=3pt, shorten >=3pt] (ab) to (ba);
        \draw [->, shorten <=3pt, shorten >=3pt] (ac) to (bb);
        \draw [->, shorten <=3pt, shorten >=3pt] (ad) to (bb);
        \draw [->, shorten <=3pt, shorten >=3pt] (ae) to (bc);
        \draw [->, shorten <=3pt, shorten >=3pt] (af) to (bc);
        \draw [->, shorten <=3pt, shorten >=3pt] (ag) to (bd);
        \draw [->, shorten <=3pt, shorten >=3pt] (ah) to (bd);
        \draw [->, shorten <=3pt, shorten >=3pt] (ai) to (be);
        \draw [->, shorten <=3pt, shorten >=3pt] (aj) to (be);
        
        \draw [->, shorten <=3pt, shorten >=3pt] (ba) to (ca);
        \draw [->, shorten <=3pt, shorten >=3pt] (bb) to (ca);
        \draw [->, shorten <=3pt, shorten >=3pt] (bc) to (cb);
        \draw [->, shorten <=3pt, shorten >=3pt] (bd) to (cb);
        
        \draw [->, shorten <=3pt, shorten >=3pt] (cb) to (db);
        \draw [->, shorten <=3pt, shorten >=3pt] (be) to (db);
        
        \draw [->, shorten <=3pt, shorten >=3pt] (ca) to (da);
        \draw [->, shorten <=3pt, shorten >=3pt] (db) to (da);
        
        
        \node (startbrace2) [node distance = 10 pt, below right of = aj, yshift=10pt] {};
        \node (stopbrace2) [node distance = 3.4 cm, below of = startbrace2] {};

        \draw [decorate,decoration={brace,amplitude=10pt,raise=4pt}]
              (startbrace2) -- (stopbrace2) node [black,midway,xshift=1.17cm] {Reduce};
      }
      \only<3->{
        \node (aa)  [below of = a, node distance = 32 pt, draw,circle, minimum size=8pt]{};
        \node (ab)  [right of = aa, node distance = 16 pt, draw,circle, minimum size=8pt]{};
        \node (ac)  [right of = ab, node distance = 16 pt, draw,circle, minimum size=8pt]{};
        \node (ad)  [right of = ac, node distance = 16 pt, draw,circle, minimum size=8pt]{};
        \node (ae)  [right of = ad, node distance = 16 pt, draw,circle, minimum size=8pt]{};
        \node (af)  [right of = ae, node distance = 16 pt, draw,circle, minimum size=8pt]{};
        \node (ag)  [right of = af, node distance = 16 pt, draw,circle, minimum size=8pt]{};
        \node (ah)  [right of = ag, node distance = 16 pt, draw,circle, minimum size=8pt]{};
        \node (ai)  [right of = ah, node distance = 16 pt, draw,circle, minimum size=8pt]{};
        \node (aj)  [right of = ai, node distance = 16 pt, draw,circle, minimum size=8pt]{};
      }
      \only<3->{
        \node (ba)  [below of = aa, node distance = 24 pt, xshift=8pt, draw, circle, minimum size=8pt]{};
        \node (bb)  [below of = ac, node distance = 24 pt, xshift=8pt, draw, circle, minimum size=8pt]{};
        \node (bc)  [below of = ae, node distance = 24 pt, xshift=8pt, draw, circle, minimum size=8pt]{};
        \node (bd)  [below of = ag, node distance = 24 pt, xshift=8pt, draw, circle, minimum size=8pt]{};
        \node (be)  [below of = ai, node distance = 24 pt, xshift=8pt, draw, circle, minimum size=8pt]{};
        
        \node (ca)  [below of = ba, node distance = 24 pt, xshift=16pt, draw, circle, minimum size=8pt]{};
        \node (cb)  [below of = bc, node distance = 24 pt, xshift=16pt, draw, circle, minimum size=8pt]{};
        
        \node (da)  [below of = ca, node distance = 36 pt, xshift=48pt, draw, circle, minimum size=8pt]{};
        \node (db)  [below of = bd, node distance = 36 pt, xshift=16pt, draw, circle, minimum size=8pt]{};
        
        \draw [->, shorten <=3pt, shorten >=3pt] (aa) to (ba);
        \draw [->, shorten <=3pt, shorten >=3pt] (ab) to (ba);
        \draw [->, shorten <=3pt, shorten >=3pt] (ac) to (bb);
        \draw [->, shorten <=3pt, shorten >=3pt] (ad) to (bb);
        \draw [->, shorten <=3pt, shorten >=3pt] (ae) to (bc);
        \draw [->, shorten <=3pt, shorten >=3pt] (af) to (bc);
        \draw [->, shorten <=3pt, shorten >=3pt] (ag) to (bd);
        \draw [->, shorten <=3pt, shorten >=3pt] (ah) to (bd);
        \draw [->, shorten <=3pt, shorten >=3pt] (ai) to (be);
        \draw [->, shorten <=3pt, shorten >=3pt] (aj) to (be);
        
        \draw [->, shorten <=3pt, shorten >=3pt] (ba) to (ca);
        \draw [->, shorten <=3pt, shorten >=3pt] (bb) to (ca);
        \draw [->, shorten <=3pt, shorten >=3pt] (bc) to (cb);
        \draw [->, shorten <=3pt, shorten >=3pt] (bd) to (cb);
        
        \draw [->, shorten <=3pt, shorten >=3pt] (cb) to (db);
        \draw [->, shorten <=3pt, shorten >=3pt] (be) to (db);
        
        \draw [->, shorten <=3pt, shorten >=3pt] (ca) to (da);
        \draw [->, shorten <=3pt, shorten >=3pt] (db) to (da);
      } 
      \only<1>{ 
        \node (startbrace2) [node distance = 10 pt, below right of = aj, yshift=10pt] {};
        \node (stopbrace2) [node distance = 3.4 cm, below of = startbrace2] {};
      }
      \uncover<3->{
        \draw [decorate,decoration={brace,amplitude=10pt,raise=4pt}]
              (startbrace1) -- (stopbrace2) node (mapreduce) [black,midway,xshift=2.7cm] {Map followed by reduce};

      }
      \uncover<4>{
            \node [node distance = 36 pt, below of = mapreduce] {
                \begin{minipage}{0.3\textwidth}
                    \textcolor{darkpastelgreen}{(Of course you don't \textit{have to} finish with reduce)}
                \end{minipage}
                };
      }
    \end{tikzpicture}
        
    \end{block}
    \end{frame}

    \begin{frame}
        \frametitle{Map and Reduce}
        \framesubtitle{Map and Reduce}
        \begin{block}{Learning to work with map and reduce}
            \begin{itemize}
                \item Map and reduce can be used with \alert{many languages} so it is a \alert{general skill} worth learning
                \item Just make sure that you \alert{know what you have in your list!}
            \end{itemize}
        \end{block}
    \end{frame}

    \subsection{MapReduce}
    \begin{frame}
        \frametitle{MapReduce frameworks}
        \framesubtitle{Parallelisation}
        \begin{block}{MapReduce frameworks are parallel}
        Since each step in Map and Reduce is separate it lends itself well to
        parallelisation. The idea behind \alert{MapReduce} frameworks is to run in
        \alert{massively parallel} ways.
        \end{block}
    \end{frame}


{
    \setbeamercolor{background canvas}{bg=black}
    \setbeamercolor{normal text}{fg=white}\usebeamercolor*{normal text}
    \setbeamercolor{item}{fg=white, bg=black}
    
    \begin{frame}[plain]
        \begin{center}
        Let's take a step back,
        \end{center}
    \end{frame}
    
    \setbeamertemplate{background}{%
    \parbox[c][\paperheight]{\paperwidth}{%
        \vskip -0.1 ex \hskip -1 em
        \includegraphics[width=1.05\paperwidth]{Figures/question-mark.jpg}
    }   
    }

    \begin{frame}[plain]
        \begin{center}
            \vspace{0.65\textheight}
            \Huge \textbf{Why are we doing this?}
        \end{center}
    \end{frame}
    \setbeamertemplate{background}{%
        \parbox[c][\paperheight]{\paperwidth}{%
            \vskip -0.1 ex \hskip -1 em
            \includegraphics[width=1.05\paperwidth]{Figures/uppmax.png}
        } 
    }
    \begin{frame}[plain]
        \begin{center}
            \Huge Parallel computing! \pause
            \vspace{0.15\textheight}

            \Large These days even the new iPhone has 6 cores\ldots 
            
            \vspace{0.05\textheight}

            \normalsize and when you have a big data you want to\pause
            
            \vspace{0.1\textheight}
            
            \Huge run in parallel!
        \end{center}
    \end{frame}
}
       
    \begin{frame}
        \frametitle{MapReduce frameworks}
        \framesubtitle{Parallelisation}
        \begin{block}{Apache Hadoop}
        \begin{wrapfigure}[2]{r}{0.45\textwidth}
        \vspace{-1.2\baselineskip}
        \includegraphics[width=1\linewidth]{figures/hadoop.png}
        \end{wrapfigure}
        Apache Hadoop is an example of a MapReduce framework that:
        \begin{itemize}
            \item Was \alert{released in 2006}
            \item Works by loading things \alert{from disk}, performing map and reduce operations and then \alert{writing back to disk}.
            \item Can use relatively \alert{cheap computers} to run things in \alert{parallel}
            \item Has relatively \alert{hard APIs to code against} while doing the map and reduce operations
        \end{itemize}
        \end{block}
    \end{frame}

    \begin{frame}
        \frametitle{MapReduce frameworks}
        \framesubtitle{Parallelisation}
        \centering
        In the Spark laboration we will load a file from the Hadoop file system:
        \begin{block}{Hadoop File System (HDFS)}
            The Hadoop File system (HDFS) is a \alert{distributed},
            \alert{scalable} storage solution designed for large data.
        \end{block}
    \end{frame}
    


\section{Spark}
    \begin{frame}
        \frametitle{Spark}
        \framesubtitle{``a unified analytics engine for large-scale data processing''}
        \begin{block}{Apache Spark}
        \begin{wrapfigure}[3]{r}{0.35\textwidth}
        \vspace{-1\baselineskip}
        \includegraphics[width=0.9\linewidth]{figures/Spark.png}\hfill
        \end{wrapfigure}
        In \alert{2014} Spark was released, in a way as a sort of response to Hadoop. Spark:
        \begin{itemize}
            \item Keeps things \alert{in memory} and is thus faster than Hadoop
        \end{itemize}\vspace{-4pt}

        \begin{itemize}
            \item Has \alert{friendly API}s for Python, Scala and to some degree R
            \item Has \alert{better support for machine learning} than Hadoop
            \item Requires \alert{relatively expensive computers} to run because it
            needs huge amounts of RAM if it is to benefit from keeping all data
            in memory.
        \end{itemize}
        \end{block}
    \end{frame}

    \begin{frame}
        \frametitle{Spark}
        \framesubtitle{``a unified analytics engine for large-scale data processing''}
        \begin{block}{Apache Spark}
        \begin{center}
            \only<1>{\includegraphics[width=0.9\textwidth]{figures/spark-overview.png}}
            \only<2>{\includegraphics[width=0.9\textwidth]{figures/spark-overview-select.png}}
        \end{center}
        \end{block}
    \end{frame}

    \subsection{Architecture}
        \begin{frame}
            \frametitle{Spark}
            \framesubtitle{Architecture}
            \begin{block}{Spark Architecture}
            \begin{center}
                \includegraphics[width=0.9\textwidth]{spark-architecture.png}
            \end{center}
            \end{block}

        \end{frame}

    \subsection{RDD}
        \begin{frame}
            \frametitle{Spark}
            \framesubtitle{RDD -- ``A read-only collection of records partitioned through the nodes in the cluster''}
            \begin{block}{RDD -- Resilient Distributed Dataset}
                When coding in Spark you work with RDDs. They are:
                \begin{itemize}
                    \item \alert{Immutable}
                    \item \alert{Distributed} and \alert{partitioned} in the cluster
                    \item \alert{Scalable} and \alert{fault tolerant}
                    \item Created from stored data or by transforming another RDD
                \end{itemize}
            \end{block}
        \end{frame}

        \begin{frame}
            \frametitle{Spark}
            \framesubtitle{RDD -- ``A read-only collection of records partitioned through the nodes in the cluster''}
            \begin{block}{Working with RDDs}
                \begin{itemize}
                    \item Spark can drop RDD's out of memory and they can then be \alert{recomputed at any time} as needed
                    \item We will be working with Python in this course, but
                    since Spark is written in Scala error messages will be \alert{Java
                    errors} and naming conventions are \alert{Java naming}. \alert{Don't worry about that!}
                \end{itemize}
            \end{block}
        \end{frame}

        
        \begin{frame}
            \frametitle{Spark}
            \framesubtitle{RDD -- ``A read-only collection of records partitioned through the nodes in the cluster''}
            \begin{block}{Transformations}
                A transformation 
                \begin{itemize}
                    \item creates a \alert{new RDD} from an old RDD
                    \item is evaluated \alert{lazily}
                \end{itemize}
            \end{block}
            \begin{block}{Actions}
                An action 
                \begin{itemize}
                    \item \alert{saves data} to file system or \alert{returns data} to driver program
                    \item \alert{blocks} while being evaluated
                \end{itemize}
            \end{block}
        \end{frame}

    \subsection{Dataframe}
        \begin{frame}
            \frametitle{Spark}
            \framesubtitle{Dataframe}
            \begin{block}{Dataframe}
                When doing machine learning with Spark we will work a lot with
                dataframes. A dataframe consists of rows and columns.

                A bit like working with tables in Microsoft Excel \smiley{} 
            \end{block}
        \end{frame}

\section{Example}
        \begin{frame}
            \frametitle{Spark}
            \framesubtitle{Example}
            \begin{center}
                Now, let's look at some examples!
            \end{center}
        \end{frame}

\subsection{Word counting (RDD)}
    \begin{frame}[fragile]
        \frametitle{Example}
        \framesubtitle{Word counting (RDD)}
        \begin{center}

        \begin{minipage}{0.8\textwidth}
\begin{lstlisting}[escapechar={|}, title=wordcount.py]
|\only<1>{\color{black}}{\only<2->{\color{gray}}from pyspark.sql import SparkSession |

|\only<1>{\color{black}}{\only<2->{\color{Orchid}}spark = SparkSession.builder.appName("SimpleApp").getOrCreate()|
|\only<1>{\color{black}}{\only<2->{\color{Orchid}}sc = spark.sparkContext|

|\only<1>{\color{black}}{\only<2->{\color{NavyBlue}}rdd0 = sc.textFile("foo.txt")|

|\only<1>{\color{black}}{\only<2->{\color{NavyBlue}}rdd1 = rdd0.flatMap( lambda line : line.split(" ") ) \only<3>{\commentfont \color{darkpastelgreen} \# transformation}|
|\only<1>{\color{black}}{\only<2->{\color{NavyBlue}}rdd2 = rdd1.map( lambda word : (word,1) ) \only<3>{\commentfont \color{darkpastelgreen} \# transformation}| 
|\only<1>{\color{black}}{\only<2->{\color{NavyBlue}}rdd3 = rdd2.reduceByKey( lambda a,b : (a + b) ) \only<3>{\commentfont \color{darkpastelgreen} \# transformation}|

|\only<1>{\color{black}}{\only<2->{\color{BrickRed}}print(rdd3.collect()) \only<3>{\commentfont \color{darkpastelgreen} \# action}|
\end{lstlisting}
\end{minipage}
\end{center}
    \end{frame}

    \begin{frame}[plain]
    \hspace*{-1em}
            \begin{tikzpicture}[node distance= 25 pt]
        \only<1>{
        \node (txt)  [fill={rgb:purple,1;white,4}, draw,minimum height=.5cm,minimum width = .5cm, node distance = 55 pt]{foo bar foo bar buzz};
        \node [above of = txt, node distance = 14 pt] {\small\commentfont foo.txt };
        }

        \only<6>{
        \node (txt)  [fill={rgb:purple,1;white,4}, draw,minimum height=.5cm,minimum width = .5cm, node distance = 55 pt]{foo bar foo bar buzz};
        \node [above of = txt, node distance = 14 pt] {\small\commentfont foo.txt };
        }
        
        \uncover<2-5>{
        \node (txt)  [fill={rgb:purple,1;white,15}, draw,minimum height=.5cm,minimum width = .5cm, node distance = 55 pt]{foo bar foo bar buzz};
        \node [above of = txt, node distance = 14 pt] {\small\commentfont foo.txt };
        }
        
        \only<2->{
        \node (rdd0)  [fill={rgb:purple,1;white,4}, below of = txt, draw,minimum height=.5cm,minimum width = .5cm, node distance = 55 pt]{"foo bar foo bar buzz"};
        \node [above of = rdd0, node distance = 14 pt] {\small\commentfont rdd0 };
        \draw [->, shorten <=4pt, shorten >=4pt,out=-35, in=35] (txt) to node[right]
              {\color{NavyBlue}\small\texttt{rdd0 = sc.textFile("foo.txt")}} (rdd0);
        }

        \uncover<3->{
        \node (rdd0)  [fill={rgb:purple,1;white,15}, below of = txt, draw,minimum height=.5cm,minimum width = .5cm, node distance = 55 pt]{"foo bar foo bar buzz"};
        \node [above of = rdd0, node distance = 14 pt] {\small\commentfont rdd0 };
        \draw [->, shorten <=4pt, shorten >=4pt,out=-35, in=35] (txt) to node[right]
              {\color{NavyBlue!40!white}\small\texttt{rdd0 = sc.textFile("foo.txt")}} (rdd0);
        }
        
        \only<3->{
        \node (rdd1)  [fill={rgb:purple,1;white,4}, below of = rdd0, draw,minimum height=.5cm,minimum width = .5cm, node distance = 55 pt]{"foo", "bar", "foo", "bar", "buzz"};
        \node [above of = rdd1, node distance = 14 pt] {\small\commentfont rdd1 };
        \draw [->, shorten <=4pt, shorten >=4pt,out=-35, in=35] (rdd0) to node[right]
              {\color{NavyBlue}\small\texttt{rdd1 = rdd0.flatMap( lambda line : line.split(" ") )}} (rdd1);
        }

        \uncover<4->{
        \node (rdd1)  [fill={rgb:purple,1;white,15}, below of = rdd0, draw,minimum height=.5cm,minimum width = .5cm, node distance = 55 pt]{"foo", "bar", "foo", "bar", "buzz"};
        \node [above of = rdd1, node distance = 14 pt] {\small\commentfont rdd1 };
        \draw [->, shorten <=4pt, shorten >=4pt,out=-35, in=35] (rdd0) to node[right]
              {\color{NavyBlue!40!white}\small\texttt{rdd1 = rdd0.flatMap( lambda line : line.split(" ") )}} (rdd1);
        }
        
        \only<4->{
        \node (rdd2)  [fill={rgb:purple,1;white,4}, below of = rdd1, draw,minimum height=.5cm,minimum width = .5cm, node distance = 55 pt]
                      {("foo", 1), ("bar", 1), ("foo", 1), ("bar", 1), ("buzz", 1)};
        \node [above of = rdd2, node distance = 14 pt] {\small\commentfont rdd2 };
        \draw [->, shorten <=4pt, shorten >=4pt,out=-35, in=35] (rdd1) to node[right]
              {\color{NavyBlue}\small\texttt{rdd2 = rdd1.map( lambda word : (word,1) )}} (rdd2);
        }
        
        \uncover<5->{
        \node (rdd2)  [fill={rgb:purple,1;white,15}, below of = rdd1, draw,minimum height=.5cm,minimum width = .5cm, node distance = 55 pt]
                      {("foo", 1), ("bar", 1), ("foo", 1), ("bar", 1), ("buzz", 1)};
        \node [above of = rdd2, node distance = 14 pt] {\small\commentfont rdd2 };
        \draw [->, shorten <=4pt, shorten >=4pt,out=-35, in=35] (rdd1) to node[right]
              {\color{NavyBlue!40!white}\small\texttt{rdd2 = rdd1.map( lambda word : (word,1) )}} (rdd2);
        }
        
        \only<5->{
        \node (rdd3)  [fill={rgb:purple,1;white,4}, below of = rdd2, draw,minimum height=.5cm,minimum width = .5cm, node distance = 55 pt]
                      {("foo", 2), ("bar", 2), ("buzz", 1)};
        \node [above of = rdd3, node distance = 14 pt] {\small\commentfont rdd3 };
        \draw [->, shorten <=4pt, shorten >=4pt,out=-35, in=35] (rdd2) to node[right]
              {\color{NavyBlue}\small\texttt{rdd3 = rdd2.reduceByKey( lambda a,b : (a + b) )}} (rdd3);
        }
        
        \uncover<6->{
        \node (rdd3)  [fill={rgb:purple,1;white,4}, below of = rdd2, draw,minimum height=.5cm,minimum width = .5cm, node distance = 55 pt]
                      {("foo", 2), ("bar", 2), ("buzz", 1)};
        \node [above of = rdd3, node distance = 14 pt] {\small\commentfont rdd3 };
        \draw [->, shorten <=4pt, shorten >=4pt,out=-35, in=35] (rdd2) to node[right]
              {\color{NavyBlue!40!white}\small\texttt{rdd3 = rdd2.reduceByKey( lambda a,b : (a + b) )}} (rdd3);
         
        \draw [->, shorten <=4pt, shorten >=4pt,out=15, in=-15, line width=1mm, opacity=0.85, color={rgb:purple,1;white,1}] (rdd3) to node[right,text width=6cm]
              {\commentfont \Large Now we have made a word count of the original file} (txt);
        }
            \end{tikzpicture}
    \end{frame}

\subsection{Dataframe}
\begin{frame}[fragile]
        \frametitle{Example}
        \framesubtitle{Dataframe}
        \vspace*{-1ex}
        \begin{center}
        \begin{minipage}{0.8\textwidth}
\begin{lstlisting}[escapechar={|}, title=dataframe.py]
|\only<1>{\color{black}}{\only<2>{\color{gray}}from pyspark.sql import SparkSession |
|\only<1>{\color{black}}{\only<2>{\color{gray}}from pyspark.sql.functions import mean |

|\only<1>{\color{black}}{\only<2>{\color{Orchid}}spark = SparkSession.builder.appName("SimpleApp").getOrCreate()|

|\only<1>{\color{black}}{\only<2>{\color{NavyBlue}}df = spark.read.format("csv")\textbackslash |
         |\only<1>{\color{black}}{\only<2>{\color{NavyBlue}}.option("header", "true")\textbackslash |
         |\only<1>{\color{black}}{\only<2>{\color{NavyBlue}}.option("inferSchema", "true").load("bar.txt")|
|\only<1>{\color{black}}{\only<2>{\color{NavyBlue!50!white}}df.show()|

|\only<1>{\color{black}}{\only<2>{\color{NavyBlue}}res = df.select([mean('foo')])|
|\only<1>{\color{black}}{\only<2>{\color{NavyBlue!50!white}}res.show()|

|\only<1>{\color{black}}{\only<2>{\color{NavyBlue}}res = df.groupBy("id").mean("foo","bar")|
|\only<1>{\color{black}}{\only<2>{\color{NavyBlue!50!white}}res.show()|

\end{lstlisting}
\end{minipage}
\end{center}
        
\end{frame}
\newsavebox{\mydfshowlisting}
\newsavebox{\myresoneshowlisting}
\newsavebox{\myrestwoshowlisting}
\begin{lrbox}{\mydfshowlisting}
\begin{lstlisting}[basicstyle=\ttfamily\tiny, frame=none]
+---+---+---+
| id|foo|bar|
+---+---+---+
|  A|  3|  0|
|  B|  1|  1|
|  B|  0|  2|
+---+---+---+
\end{lstlisting}
\end{lrbox}

\begin{lrbox}{\myresoneshowlisting}
\begin{lstlisting}[basicstyle=\ttfamily\tiny, frame=none]
+------------------+
|          avg(foo)|
+------------------+
|1.3333333333333333|
+------------------+
\end{lstlisting}
\end{lrbox}

\begin{lrbox}{\myrestwoshowlisting}
\begin{lstlisting}[basicstyle=\ttfamily\tiny, frame=none]
+---+--------+--------+
| id|avg(foo)|avg(bar)|
+---+--------+--------+
|  B|     0.5|     1.5|
|  A|     3.0|     0.0|
+---+--------+--------+
\end{lstlisting}
\end{lrbox}

    \begin{frame}[plain]
    \hspace*{-1em}
            \resizebox{!}{1.15\textheight}{
            \begin{tikzpicture}[node distance= 25 pt]
            \small
        \only<1->{
        \node (txt)  [fill={rgb:purple,1;white,4}, draw,minimum height=.5cm, text width = 1.5cm, node distance = 55 pt]
{id,foo,bar
A,3,0
B,1,1
B,0,2};
        \node [above of = txt, node distance = 31 pt] {\footnotesize\commentfont bar.txt };
        }
        \uncover<2->{
        \node (txt)  [fill={rgb:purple,1;white,15}, draw,minimum height=.5cm, text width = 1.5cm, node distance = 55 pt]
{id,foo,bar
A,3,0
B,1,1
B,0,2};
        \node [above of = txt, node distance = 31 pt] {\footnotesize\commentfont bar.txt };
        }

        \only<2->{
        \node (load) [right of = txt, node distance = 7 cm, text width = 10cm]
              {\begin{minipage}{10cm}
\color{NavyBlue}\small\texttt
{ \begin{tabular}{r@{\hskip0pt}l}
df = spark&.read.format("csv")\textbackslash \\
          &.option("header", "true")\textbackslash \\
          &.option("inferSchema", "true").load("bar.txt") \\
   \end{tabular}}
              \end{minipage}
          };
        }
        \uncover<3->{
        \node (load) [right of = txt, node distance = 7 cm, text width = 10cm]
              {\begin{minipage}{10cm}
\color{NavyBlue!40!white}\small\texttt
{ \begin{tabular}{r@{\hskip0pt}l}
df = spark&.read.format("csv")\textbackslash \\
          &.option("header", "true")\textbackslash \\
          &.option("inferSchema", "true").load("bar.txt") \\
   \end{tabular}}
              \end{minipage}
          };
        }
        \only<3->{
        \node (select) [below of = load, node distance = 3.5cm] 
              {\color{NavyBlue}\small\texttt{ res1 = df.select([mean('foo')]) }};
        }
        \uncover<4->{
        \node (select) [below of = load, node distance = 3.5cm] 
              {\color{NavyBlue!40!white}\small\texttt{ res1 = df.select([mean('foo')]) }};
        }
  

        \only<2->{
        \draw [->, shorten <=4pt, shorten >=4pt,out=-35, in=35] (load) to node (dfshow) [fill=white]
              {\color{NavyBlue!40!white}\small\texttt{df.show()}} (select);

        \node (dfshowT)  [right of = dfshow, fill={rgb:purple,1;white,4}, draw,minimum height=.5cm, text width = 1.8cm, node distance = 65 pt]
        {\usebox\mydfshowlisting};
        \node [above of = dfshowT, node distance = 33 pt] {\footnotesize\commentfont df };
        }

        \uncover<3->{
        \draw [->, shorten <=4pt, shorten >=4pt,out=-35, in=35] (load) to node (dfshow) [fill=white]
              {\color{NavyBlue!20!white}\small\texttt{df.show()}} (select);

        \node (dfshowT)  [right of = dfshow, fill={rgb:purple,1;white,15}, draw,minimum height=.5cm, text width = 1.8cm, node distance = 65 pt]
        {\usebox\mydfshowlisting};
        \node [above of = dfshowT, node distance = 33 pt] {\footnotesize\commentfont df };
        }

        \uncover<4->{
        \node (groupby) [below of = select, node distance = 3cm] 
              {\color{NavyBlue}\small\texttt{ res2 = df.groupBy("id").mean("foo","bar") }};
        }
        \only<3->{
        \draw [->, shorten <=4pt, shorten >=4pt,out=-35, in=35] (select) to node (res1show) [fill=white]
              {\color{NavyBlue!40!white}\small\texttt{res1.show()}} (groupby);
        }
        \uncover<4->{
        \draw [->, shorten <=4pt, shorten >=4pt,out=-35, in=35] (select) to node (res1show) [fill=white]
              {\color{NavyBlue!20!white}\small\texttt{res1.show()}} (groupby);
        }
        \only<3->{
        \node (res1showT)  [right of = res1show, fill={rgb:purple,1;white,4}, draw,minimum height=.5cm, text width = 2.5cm, node distance = 75 pt]
        {\usebox\myresoneshowlisting};
        \node [above of = res1showT, node distance = 26 pt] {\footnotesize\commentfont res1 };
        }
        \uncover<4->{
        \node (res1showT)  [right of = res1show, fill={rgb:purple,1;white,15}, draw,minimum height=.5cm, text width = 2.5cm, node distance = 75 pt]
        {\usebox\myresoneshowlisting};
        \node [above of = res1showT, node distance = 26 pt] {\footnotesize\commentfont res1 };
        }

        \uncover<4->{
        \node (empty) [below of = groupby, node distance = 2.6cm]{};
        \draw [->, shorten <=4pt, shorten >=4pt,out=-35, in=35] (groupby) to node (res2show) [fill=white]
              {\color{NavyBlue!40!white}\small\texttt{res2.show()}} (empty);

        \node (res2showT)  [right of = res2show, fill={rgb:purple,1;white,4}, draw,minimum height=.5cm, text width = 2.8cm, node distance = 80 pt]
        {\usebox\myrestwoshowlisting};
        \node [above of = res2showT, node distance = 29 pt] {\footnotesize\commentfont res2 };
        }
            \end{tikzpicture}}
    \end{frame}

    \subsection{Some commands}
    \begin{frame}
        \frametitle{Example}
        \framesubtitle{Some commands}
        I will now give examples of some commands, these lists are in no way complete!
    \end{frame}

    \begin{frame}
        \frametitle{Example}
        \framesubtitle{Some commands}
        \oldBlock{RDD's}
            \begin{center}
            \begin{tabular}{ll}
            \toprule
            Command & Explanation \\
            \midrule
            \texttt{.take(n)} & Bring $n$ things from list to local Python environment \\
            \texttt{.collect()} & Bring all back to local, \alert{Don't do on big RDD's!} \\
            \texttt{.parallelize} & Put things on cluster from local \\
            \texttt{.count()} & Number of items in the list \\
            \texttt{.flatmap} & flatten structure ([[1,2],[3,4]] $\Rightarrow$ [1,2,3,4]) after applying a map \\
            \texttt{.reduceByKey} & merge values by key during reduce \\
            \bottomrule
            \end{tabular}
            \end{center}
        \oldEndBlock{}
    \end{frame}
    
    \begin{frame}
        \frametitle{Example}
        \framesubtitle{Some commands}
        \oldBlock{Dataframes}
            \begin{center}
            \begin{tabular}{ll}
            \toprule
            Command & Explanation \\
            \midrule
            \texttt{.show()} & Print a representation of the dataframe \\
            \texttt{.printSchema()} & Print datatypes and other info about a dataframe \\
            \texttt{.coalesce} & Reduce number of partitions a dataset is spread over \\
            \bottomrule
            \end{tabular}
            \end{center}
        \oldEndBlock{}
    \end{frame}

\setbeamertemplate{background}{%
    \parbox[c][\paperheight]{\paperwidth}{%
        \vskip -8 ex \hskip -2 em
        \includegraphics[height=1.5\paperheight]{Figures/blasippa.jpg}
    }   
    \parbox[c][\paperheight]{\paperwidth}{%
        \vskip 25 ex \hskip -40 em
        \color{white}\fbox{\includegraphics[height=0.37\paperheight]{Figures/me.jpg}}
    }   
}
\begin{frame}[plain]
    \vfill\hfill{\Huge\qquad\color{white} \zB Thank \zC you}\hfill\hfill\hfill\vfill
\end{frame}
\setbeamertemplate{background}{}
\end{document}
