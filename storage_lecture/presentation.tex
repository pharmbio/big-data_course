\documentclass[aspectratio=169,usenames,dvipsnames]{beamer}

\usepackage{pgf}  
\usepackage{tikz}
\usetikzlibrary{arrows}
\usepgflibrary{shapes.arrows} 
\usetikzlibrary{intersections}
\usetikzlibrary{calc}
\usetikzlibrary{fit}
\usetikzlibrary{automata,positioning}
\usepackage{pgfplots,stackengine}
\usepackage{fontspec}
\usepackage{fancyvrb}
\usepackage{wasysym}
\usepackage{unicode-math}
\usepackage{import}
\usepackage{rotating}
\usepackage{gensymb}
\usepackage{chemfig}
\usepackage{rotating}
\usepackage{booktabs}
\usepackage{pifont}
\usepackage{wrapfig}
\usepackage{mathtools}
\usepackage{graphbox}
\usepackage{epigraph}
\usepackage{listings}
\usepackage{verbatim}
\usepackage{hologo}
\usepackage{wrapfig}
\usepackage[absolute,overlay]{textpos}
\usepackage[euler-digits,euler-hat-accent]{eulervm}
%\logo{\pgfputat{\pgfxy(.45,.5)}{\pgfbox[center]{\includegraphics[width=1.7cm]{Figures/uu_shadow.pngu}}}}

\usetheme{Copenhagen}
\usecolortheme{beaver}

\definecolor{uured}{RGB}{153,0,0}
\setbeamercolor{block title}{use=structure,fg=white,bg=uured}
\setbeamercolor*{item}{fg=red}

\newcommand{\unilogo}{
  \setlength{\TPHorizModule}{1pt}
  \setlength{\TPVertModule}{1pt}
  \begin{textblock}{1}(26,-10)
   \includegraphics[height=70pt, align=c]{Figures/uu_shadow.png}
  \end{textblock}
  } 

\pgfmathdeclarefunction{gauss}{2}{%
  \pgfmathparse{1/(#2*sqrt(2*pi))*exp(-((x-#1)^2)/(2*#2^2))}%
}
  
\makeatletter
    \newcases{mycases}{\quad}{%
        \hfil$\m@th\displaystyle{##}$}{$\m@th\displaystyle{##}$\hfil}{\lbrace}{.}
\makeatother

\addtobeamertemplate{frametitle}{}{%
    \unilogo
}
\LetLtxMacro{\oldBlock}{\block}
\LetLtxMacro{\oldEndBlock}{\endblock}
\renewcommand{\block}{\begin{center}\begin{minipage}{0.8\textwidth}\oldBlock}
\renewcommand{\endblock}{\oldEndBlock\end{minipage}\end{center}}
\definecolor{darkpastelgreen}{rgb}{0.01, 0.75, 0.24}

\setlength{\fboxsep}{0pt}

\begin{document}
\graphicspath{{Figures/}}
\setsansfont[ItalicFont = Optima Italic,
             BoldFont = Optima Bold,
             Ligatures=TeX ]
            {Optima Regular}
\setmainfont[ItalicFont = Optima Italic,
             BoldFont = Optima Bold,
             Ligatures=TeX]
            {Optima Regular}
\newfontfamily\commentfont[]{Chalkboard}
\newfontfamily\DejaSans{DejaVuSans.ttf}
\newfontfamily\herculanum[]{Herculanum}
\newfontfamily\timesfont[ItalicFont = Times New Roman Italic]{Times New Roman}
\newcommand{\lmr}{\fontfamily{lmr}\selectfont}
\newfontfamily\zA[Ligatures={Common, Rare}, Variant=1] {Zapfino}
\newfontfamily\zB[Ligatures={Common, Rare}, Variant=2] {Zapfino}
\newfontfamily\zC[Ligatures={Common, Rare}, Variant=3] {Zapfino}
\newfontfamily\zD[Ligatures={Common, Rare}, Variant=4] {Zapfino}
\newfontfamily\zE[Ligatures={Common, Rare}, Variant=5] {Zapfino}
\newfontfamily\zF[Ligatures={Common, Rare}, Variant=6] {Zapfino}
\newfontfamily\zG[Ligatures={Common, Rare}, Variant=7] {Zapfino}
\renewcommand\UrlFont{\color{blue}}
\renewcommand\thefootnote{\textcolor{uured}{\arabic{footnote}}}
\setbeamercolor{alerted text}{fg=uured}
\lstset{basicstyle=\ttfamily\scriptsize, frame=single }
\newcommand{\TikZ}{{\lmr Ti\textit{k}Z}}

\title{Big data storage and transfer}   
\author{Jonathan Alvarsson} 
%\titlegraphic{\vfill\includegraphics[width=18em]{Figures/ORN_large.png}}
\date{\today} 

\setbeamertemplate{background}{%
    \parbox[c][\paperheight]{\paperwidth}{%
        \vfill
        \hfill
        \includegraphics[height=0.65\textheight]{Figures/sigill.png}
    }   
}
\begin{frame}[plain]
\unilogo \vspace{1cm} \titlepage
\begin{tikzpicture}[remember picture,overlay]
\tikz[remember picture, overlay] \fill[uured] (current page.north west) rectangle ++(\paperwidth,-0.5cm);
\end{tikzpicture}%
\end{frame}

\setbeamertemplate{background}{}
\renewcommand{\unilogo}{
  \setlength{\TPHorizModule}{1pt}
  \setlength{\TPVertModule}{1pt}
  \begin{textblock}{1}(0,0)
   \includegraphics[height=27pt, align=c]{Figures/uu.png}
  \end{textblock}
  } 
    \begin{frame}
    \frametitle{Outline}
    \begin{minipage}{0.25\textwidth}
    \mbox{}
    \end{minipage}
    \begin{minipage}{0.6\textwidth}
    \tableofcontents[hideallsubsections]
    \end{minipage}
    \end{frame}

\section{Transferring data}
    \subsection{SFTP, wget, scp and rsync}
    \begin{frame}
        \frametitle{Transferring data}
        \framesubtitle{SFTP, wget, scp and rsync}
        \begin{block}{File Transfer Protocoll -- FTP}
        The FTP standard:
        \begin{itemize}
            \item was \alert{published in 1971}
            \item was not designed be secure and has \alert{many security issues}
            \item is \alert{no longer supported by Chrome and Firefox} 
        \end{itemize}
        \end{block}
    \end{frame}
    \begin{frame}
        \frametitle{Transferring data}
        \framesubtitle{SFTP, wget, scp and rsync}
        \begin{block}{SSH File Transfer Protocol -- SFTP}
        The SFTP (sometimes called Secure File Transfer Protocol) standard:
        \begin{itemize}
            \item is not FTP over SSH but a separate protocol
            \item is about the file transferring, SSH takes care of the security 
        \end{itemize}
        \end{block}
    \end{frame}
    \begin{frame}
        \frametitle{Transferring data}
        \framesubtitle{sftp, wget, scp and rsync}
        \begin{block}{wget}
            wget 
            \begin{itemize}
                \item appeared in 1996
                \item derives it name from World Wide Web and get
                \item is a program to retrieve data from web servers
                \item supports HTTP, HTTPS and FTP
            \end{itemize}
        \end{block}
    \end{frame}
    \begin{frame}
        \frametitle{Transferring data}
        \framesubtitle{sftp, wget, scp and rsync}
        \begin{block}{checksums}
            Although wget is designed to be robust it is often a good idea to
            check it with md5sum after transfer. The md5 hash functions like 
            a digital fingerprint that you can calculate for a file. After
            downloading a big file from a web server you \alert{compute the hash and
            compare with what it is supposed to be} in order to see that your
            file was transferred correctly.
        \end{block}
    \end{frame}
    \begin{frame}
        \frametitle{Transferring data}
        \framesubtitle{sftp, wget, scp and rsync}
        \begin{block}{Secure copy protocol -- scp}
        scp:
        \begin{itemize}
            \item commonly refers to both the protocol and the software,
            \item has a \alert{syntax very similar to the common cp command}
            \item is considered \alert{outdated, inflexible} and not readily fixed by
            the developers. sftp or rsync is recommended instead.
        \end{itemize}
        \end{block}
    \end{frame}
    \begin{frame}
        \frametitle{Transferring data}
        \framesubtitle{sftp, wget, scp and rsync}
        \begin{block}{rsync}
        rsync
        \begin{itemize}
            \item appeared in 1996
            \item is a program for \alert{transferring} and \alert{synchronising} files
            \item can when synchronising files \alert{send only the changes} of a file 
            \item \alert{uses md5 checksums} to make sure transferred files are identical to original
        \end{itemize}
        \end{block}
    \end{frame}
    \subsection{Hard drives}
    \begin{frame}
        \frametitle{Transferring data}
        \framesubtitle{Hard drives}
        \begin{block}{Hard drives}
            
        \end{block}
    \end{frame}
    \subsection{Commercial solutions}
    \begin{frame}
        \frametitle{Transferring data}
        \framesubtitle{Commercial solution}
        \begin{block}{Commercial solutions}
            IBM Aspera
        \end{block}
    \end{frame}

\section{Compression}
    \subsection{zip}
    \begin{frame}
        \frametitle{Compression}
        \framesubtitle{zip}
        \begin{block}{zip}
        \end{block}
    \end{frame}
    \subsection{gz and tar.gz}
    \begin{frame}
        \frametitle{Compression}
        \framesubtitle{gz and tar.gz}
        \begin{block}{gz and tar.gz}
        \end{block}
    \end{frame}

\section{Storage when computing}
    \subsection{Local storage / Fast storage}
    \begin{frame}
        \frametitle{Storage when computing}
        \framesubtitle{Local storage / Fast storage}
        \begin{block}{Local storage}
        \end{block}
    \end{frame}
    \subsection{Object storage S3}
    \begin{frame}
        \frametitle{Storage when computing}
        \framesubtitle{Object storage}
        \begin{block}{bject storage}
        \end{block}
    \end{frame}
    \subsection{Hadoop FS?}
    \begin{frame}
        \frametitle{Storage when computing}
        \framesubtitle{Hadoop FS?}
        \begin{block}{Should we talk about this?}
        \end{block}
    \end{frame}

\setbeamertemplate{background}{%
    \parbox[c][\paperheight]{\paperwidth}{%
        \vskip -8 ex \hskip -2 em
        \includegraphics[height=1.5\paperheight]{Figures/blasippa.jpg}
    }   
    \parbox[c][\paperheight]{\paperwidth}{%
        \vskip 25 ex \hskip -40 em
        \color{white}\fbox{\includegraphics[height=0.37\paperheight]{Figures/me.jpg}}
    }   
}
\begin{frame}[plain]
    \vfill\hfill{\Huge\qquad\color{white} \zB Thank \zC you}\hfill\hfill\hfill\vfill
\end{frame}
\setbeamertemplate{background}{}
\end{document}
